\chapter{Motivation}

%	bpmn introduce
	Business processes are fundamental elements for companies and organizations. They aggregate all the tasks, activities, and timelines involved in companies' workflow whose aim is to provide business or to create value \cite{literature_review_2}. Business Process Modeling Notation, also known as BPMN is a modeling language describing such workflows by using graphical notations and thus provides an easily understandable overview of the operations performed in the organization for all business users \cite{literature_review_1}. 
	
	Due to the importance of Business processes, leveraging the BPMN techniques can positively affect an organization's performance and thus increase its productivity. However, not everyone is familiar with the BPMN designing techniques. Consequently, managers, along with other process participants prefer using natural language to define business processes. As a result, organizations usually have a large amount of information stored as text documents \cite{literature_review_2}. There is a need to translate text documents into the process model regarding such a situation. However, process modeling is not a simple task, but is time-consuming and experts with professional knowledge are required. 
%todo: if there is a need to expand the details of difficulties of process modeling, then refer to literature_review_2 
	
%	nlp introduce
	Over the past years, the development of AI techniques brought solutions to many technical difficulties. Natural Language Processing (NLP), as one of the AI's branches, could possibly address the problem of the difficulties in process modeling. Natural Language Processing is an interdisciplinary discipline focusing on the study of algorithms that enable the computer to understand and process the human language\cite{t2m_3}. During the understanding and processing of the natural language text, NLP performs three types of analysis: Firstly, morphological analysis is performed, which analyze the structure of words. The syntactic analysis then explores the grammar relationship between words in sentences, deciding which grammar category the word belongs to. Finally, semantic analysis is executed, which leverages the afore analyses to define the meaning of the text based on the knowledge of sentence structure and the relationship between words \cite{literature_review_2}. 
%todo: if there is a need to expand the details of NLP analysis, then refer to literature_review_2 introduction part
	
%	why should nlp be used to generate the bpmn
	The unique features of the NLP technique make it very suitable for exploiting information from the text documents that record the firm's business process and then analyzing the data to generate the process models automatically. This paper serves as a proposal to suggest using NLP to extract the information from text written in nature language and automatically generate the corresponding business model.
	
	\section{Research Questions}

	The main research question (\textbf{RQ}) is formulated as: "\textit{How can business process models be generated from regulatory documents automatically using the Nature language processing technique?}". To better answer the main research question, three embedded aspects can be revealed: \textbf{RQ1}: "\textit{which NLP methods can be used to extract information?}"; \textbf{RQ2}: "\textit{How can the extracted information be analyzed and composed to generate business process models?}" and finally \textbf{RQ3}: "\textit{How does the proposed approach perform with different kinds of input documents?}"

	
	Currently, there exist various tools, libraries, and dependencies for NLP. Therefore, the first research question \textbf{RQ1} tries to figure out which methods are the most suitable ones to use to extract information from regulatory documents. The methods should be able to separate sentences, label each word in a sentence with corresponding syntactical tags, and analyze the grammatical relationships between words. By doing so, we are able to explore the information hidden behind the natural language and thus use them for further operation. In the next step, \textbf{RQ2} explores how to use syntactical and grammatical information to determine events of business processes, identify the conditional restraints ("and" or "or"), and the sequential orders of business processes. Once such information is acquired, an algorithm should be developed to combine all the business processes in a logical order. In the end, the composed process model should be well visualized. The last research question \textbf{RQ3} tries to discover the adaptability of the proposed model: How well does the method perform with the document other than the regulatory document? Does the accuracy of the outcome decrease with other kinds of documents? A sets of different input documents will be prepared and a corresponding benchmark will be performed. 