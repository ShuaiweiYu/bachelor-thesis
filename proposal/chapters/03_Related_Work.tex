\chapter{Related Work}
\label{sec:works}

	In order to learn the current state-of-the-art methods of auto-generating business process models and thus answer the research question comprehensively, a systematic literature review must be performed so that we can learn what kind of efforts are made as well as what are the most preferred techniques and what open challenges exist. The literature review is conducted under the guidance of Kitchenham et al. given in their paper \cite{literature_review_guidance}. The work consists of several stages: Firstly, the electronic database used to run the search is chosen. Then the selection criteria are defined, and articles are filtered accordingly. After that, a horizontal search will be run to cover as many papers as possible. Finally, a list of the final literature is studied carefully, and helpful information is extracted.

	\begin{table}[]
		\centering
		\caption{\centering Overview of Systematic Literature Review Protocol}
		\begin{tabular}{llll}
    	\textbf{Database}\hspace{30mm} & \textbf{hits} \hspace{10mm} & \textbf{selected} &  \\
    	\hline
		IEEE                     		& 56   & 0  &      		\\
		Springer                 		& 275  & 0  &      		\\
		ACM                      		& 201  & 0  &      		\\
		Google scholar           		& 0    & 0  &      		\\
		\hline
		Result horizontal search	 	& 0    & 0  &      		\\
		Vertical search          		& 0    & 0  &  \hspace{5mm}add papers  \\
		\hline
		Overall                  		& 0    & 0  &     
		\end{tabular}
	\end{table}
	
% search string
	In order to perform a comprehensive literature review, we chose three most famous electronic databases, i.e., IEEE, Springer, and ACM. Nevertheless, only using these three databases, There is still a minor chance that some important articles will be missed. Therefore, we also used Google scholar as a complement because it covers a wide range of literature, from conference papers to degree theses. The search string used for the literature review is developed using two keywords, which are the most important ones for our research: \textit{business process model} and \textit{natural language processing}.
	
% selection criteria
	In the next step, inclusion and exclusion criteria should be defined so that the articles can be well selected to support our research and future work. 