\chapter{Discussion}
\label{sec:discussion}

From the evaluation results, our proposed system achieves good accuracy in extracting information from textual descriptions with only a little uncaptured information. Moreover, the ordering of the activities also has a relatively good precision, which makes the generated BPMN diagram in a more natural order. Nevertheless, one drawback of the proposed system is that the system generates several irrelevant information as the number of sentences within a textual description grows.

The reason why the proposed system generates irrelevant information is as discussed in previous sections: The textual description often contains many meta sentences that describe the business process as a whole concept or some example sentences explaining the described process steps. Such information benefits the human reader but makes our system inefficient when processing the textual input. It is challenging to successfully identify and filter such process irrelevant information because relevant semantic analysis is required. However, the existing natural language processing tool \textit{Spacy} does not support such semantic understanding component \cite{t2m_1_successor}.

The errors made by sentence parsing is another mentionable factor that can decrease the performance of our system. As our system is highly dependent on the parsed sentence structure, POS tagging, and dependencies. The correctness of the parsed sentence is the entry point of our system. However, due to the complexity of natural language, it is not feasible to ideally parse every sentence, and thus, errors will occur. These errors will affect the recognition precision of the rule-based system by providing incorrect tags and dependencies. An example of an incorrect parsed sentence is given in the figure \ref{img:incorrect_pasrsing}, where the latter part of the sentence is divided mistakenly into "she takes her computer home" and "unrepaired". 

\begin{figure}[h]
    \centering
    \caption{An incorrect parsed sentence}
    \label{img:incorrect_pasrsing}
    \includegraphics[width=0.8\textwidth]{tum-resources/images/discuss_01.png}
    \floatfoot{generated using https://christos-c.com/treeviewer/}
\end{figure}

Using the stop list is one of the possible solutions to identify gateways in the textual descriptions. However, many possibilities exist to express the meaning of conditional and parallel gateways. These gateways sometimes even have implicated references to previously mentioned activities. The proposed system can only capture the explicitly expressed gateways while the implications of gateways and flows still require human recognition. 

These properties suggest that the proposed system can not yet replace human modelers but can serve modelers as a complement tool for accelerating the BPMN modeling process, for the information is comprehensively extracted and placed in a logical order, modelers only need to remove the irrelevant information. Identifying the business process elements is still considered time-consuming, but removing irrelevant business processes is intuitive and straightforward. The work also suggests that the LLM model can assist with the generation of a BPMN diagram by providing adjustments and corrections to the generated results.

Using the rule-based system to extract the business process is straightforward and leads to stable output. However, the rule sets for natural language are impossible to be complete, and large rule sets can lead to a performance decrease when generating results. The recently merged Large Language Model using in-context learning shows a promising aspect in performing this task \cite{LLM_2} \cite{LLM_1}. However, the current state-of-the-art model GPT-4 output is sometimes unstable, and the model is not specifically fine-tuned for the business process model generation, which could be possible research directions in the future. 
