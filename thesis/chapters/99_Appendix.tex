\chapter{Appendix}

\section{Textual description used as input}
\label{appendix:text}
Text 1 - 15 are taken from the work \cite{t2m_1_main}, and the tokens marked with color are the PET tokens identified in \cite{pet_dataset}. Tokens marked with \textcolor{blue}{blue} stand for Actor, tokens marked with \textcolor{green}{green} stand for Activity, tokens marked with \textcolor{orange}{orange} stand for Activity Data, tokens marked with \textcolor{red}{red} stand for Gateways, tokens marked with \textcolor{brown}{brown} stand for Condition Specification, tokens marked with \textcolor{purple}{purple} stand for Further Specification. Texts 16 - 20 are taken from other sources, therefore the corresponding PET tags are labeled manually by the author of this work.

\textbf{Text 1: Process Description 1-1: Bicycle manufacturing}\\
A small company manufactures customized bicycles. Whenever \textcolor{blue}{the} \textcolor{blue}{sales} \textcolor{blue}{department} \textcolor{green}{receives} \textcolor{orange}{an} \textcolor{orange}{order}, a new process instance is created. A member of the sales department can then \textcolor{green}{reject} \textcolor{red}{or} \textcolor{green}{accept} \textcolor{orange}{the} \textcolor{orange}{order} for a customized bike. In the former case, the process instance is finished. In the latter case, \textcolor{blue}{the} \textcolor{blue}{storehouse} and \textcolor{blue}{the} \textcolor{blue}{engineering} \textcolor{blue}{department} are \textcolor{green}{informed}. \textcolor{blue}{The} \textcolor{blue}{storehouse} immediately \textcolor{green}{processes} \textcolor{orange}{the} \textcolor{orange}{part} \textcolor{orange}{list} \textcolor{orange}{of} \textcolor{orange}{the} \textcolor{orange}{order} and \textcolor{green}{checks} \textcolor{orange}{the} \textcolor{orange}{required} \textcolor{orange}{quantity} \textcolor{orange}{of} \textcolor{orange}{each} \textcolor{orange}{part}. \textcolor{red}{If} \textcolor{brown}{the} \textcolor{brown}{part} \textcolor{brown}{is} \textcolor{brown}{available} \textcolor{brown}{in-house}, \textcolor{orange}{it} is \textcolor{green}{reserved}. \textcolor{red}{If} \textcolor{brown}{it} \textcolor{brown}{is} \textcolor{brown}{not} \textcolor{brown}{available}, \textcolor{orange}{it} is \textcolor{green}{back-ordered}. This procedure is repeated for each item on the part list. \textcolor{red}{In} \textcolor{red}{the} \textcolor{red}{meantime}, \textcolor{blue}{the} \textcolor{blue}{engineering} \textcolor{blue}{department} \textcolor{green}{prepares} \textcolor{orange}{everything} for the assembling of the ordered bicycle. \textcolor{red}{If} \textcolor{brown}{the} \textcolor{brown}{storehouse} \textcolor{brown}{has} \textcolor{brown}{successfully} \textcolor{brown}{reserved} \textcolor{brown}{or} \textcolor{brown}{back-ordered} \textcolor{brown}{every} \textcolor{brown}{item} \textcolor{brown}{of} \textcolor{brown}{the} \textcolor{brown}{part} \textcolor{brown}{list} and the preparation activity has finished, \textcolor{blue}{the} \textcolor{blue}{engineering} \textcolor{blue}{department} \textcolor{green}{assembles} \textcolor{orange}{the} \textcolor{orange}{bicycle}. Afterwards, \textcolor{blue}{the} \textcolor{blue}{sales} \textcolor{blue}{department} \textcolor{green}{ships} \textcolor{orange}{the} \textcolor{orange}{bicycle} to \textcolor{blue}{the} \textcolor{blue}{customer} and finishes the process instance.


\textbf{Text 2: Process Description 1-2: Computer repair}\\
\textcolor{blue}{A} \textcolor{blue}{customer} \textcolor{green}{brings} \textcolor{green}{in} \textcolor{orange}{a} \textcolor{orange}{defective} \textcolor{orange}{computer} and \textcolor{blue}{the} \textcolor{blue}{CRS} \textcolor{green}{checks} \textcolor{orange}{the} \textcolor{orange}{defect} and \textcolor{green}{hands} \textcolor{green}{out} \textcolor{orange}{a} \textcolor{orange}{repair} \textcolor{orange}{cost} \textcolor{orange}{calculation} back. \textcolor{red}{If} \textcolor{brown}{the} \textcolor{brown}{customer} \textcolor{brown}{decides} \textcolor{brown}{that} \textcolor{brown}{the} \textcolor{brown}{costs} \textcolor{brown}{are} \textcolor{brown}{acceptable}, the process continues, \textcolor{red}{otherwise} \textcolor{blue}{she} \textcolor{green}{takes} \textcolor{orange}{her} \textcolor{orange}{computer} \textcolor{purple}{home} \textcolor{purple}{unrepaired}. The ongoing repair consists of two activities, which are executed, in an arbitrary order. The first activity is to \textcolor{green}{check} and \textcolor{green}{repair} \textcolor{orange}{the} \textcolor{orange}{hardware}, \textcolor{red}{whereas} the second activity \textcolor{green}{checks} and \textcolor{green}{configures} \textcolor{orange}{the} \textcolor{orange}{software}. After each of these activities, \textcolor{orange}{the} \textcolor{orange}{proper} \textcolor{orange}{system} \textcolor{orange}{functionality} is \textcolor{green}{tested}. \textcolor{red}{If} \textcolor{brown}{an} \textcolor{brown}{error} \textcolor{brown}{is} \textcolor{brown}{detected} \textcolor{orange}{another} \textcolor{orange}{arbitrary} \textcolor{orange}{repair} \textcolor{orange}{activity} is \textcolor{green}{executed}, \textcolor{red}{otherwise} the repair is finished.

\textbf{Text 3: Process Description 1-3: Hotel Service}\\
The Evanstonian is an upscale independent hotel. When \textcolor{blue}{a} \textcolor{blue}{guest} \textcolor{green}{calls} \textcolor{blue}{room} \textcolor{blue}{service} at The Evanstonian, \textcolor{blue}{the} \textcolor{blue}{room-service} \textcolor{blue}{manager} \textcolor{green}{takes} \textcolor{green}{down} \textcolor{orange}{the} \textcolor{orange}{order}. \textcolor{blue}{She} then \textcolor{green}{submits} \textcolor{orange}{an} \textcolor{orange}{order} \textcolor{orange}{ticket} to \textcolor{blue}{the} \textcolor{blue}{kitchen} \textcolor{purple}{to} \textcolor{purple}{begin} \textcolor{purple}{preparing} \textcolor{purple}{the} \textcolor{purple}{food}. \textcolor{blue}{She} also \textcolor{green}{gives} \textcolor{orange}{an} \textcolor{orange}{order} to \textcolor{blue}{the} \textcolor{blue}{sommelier} (i.e., the wine waiter) \textcolor{purple}{to} \textcolor{purple}{fetch} \textcolor{purple}{wine} \textcolor{purple}{from} \textcolor{purple}{the} \textcolor{purple}{cellar} \textcolor{purple}{and} \textcolor{purple}{to} \textcolor{purple}{prepare} \textcolor{purple}{any} \textcolor{purple}{other} \textcolor{purple}{alcoholic} \textcolor{purple}{beverages}. Eighty percent of room-service orders include wine or some other alcoholic beverage. Finally, \textcolor{blue}{she} \textcolor{green}{assigns} \textcolor{orange}{the} \textcolor{orange}{order} to \textcolor{blue}{the} \textcolor{blue}{waiter}. \textcolor{red}{While} \textcolor{blue}{the} \textcolor{blue}{kitchen} and \textcolor{blue}{the} \textcolor{blue}{sommelier} are \textcolor{green}{doing} \textcolor{orange}{their} \textcolor{orange}{tasks}, \textcolor{blue}{the} \textcolor{blue}{waiter} \textcolor{green}{readies} \textcolor{orange}{a} \textcolor{orange}{cart} (i.e., puts a tablecloth on the cart and gathers silverware). The waiter is also responsible for nonalcoholic drinks. Once the food, wine, and cart are ready, \textcolor{blue}{the} \textcolor{blue}{waiter} \textcolor{green}{delivers} \textcolor{orange}{it} to \textcolor{blue}{the} \textcolor{blue}{guest’s} \textcolor{blue}{room}. After \textcolor{green}{returning} \textcolor{purple}{to} \textcolor{purple}{the} \textcolor{purple}{room-service} \textcolor{purple}{station}, \textcolor{blue}{the} \textcolor{blue}{waiter} \textcolor{green}{debits} \textcolor{orange}{the} \textcolor{orange}{guest} \textcolor{orange}{’} \textcolor{orange}{s} \textcolor{orange}{account}. \textcolor{blue}{The} \textcolor{blue}{waiter} may \textcolor{green}{wait} \textcolor{orange}{to} \textcolor{orange}{do} \textcolor{orange}{the} \textcolor{orange}{billing} \textcolor{red}{if} \textcolor{brown}{he} \textcolor{brown}{has} \textcolor{brown}{another} \textcolor{brown}{order} \textcolor{brown}{to} \textcolor{brown}{prepare} \textcolor{brown}{or} \textcolor{brown}{deliver}.

\textbf{Text 4: Process Description 1-4: Underwriters}\\
\textcolor{blue}{The} \textcolor{blue}{party} \textcolor{green}{sends} \textcolor{orange}{a} \textcolor{orange}{warrant} \textcolor{orange}{possession} \textcolor{orange}{request} asking a warrant to be released. \textcolor{blue}{The} \textcolor{blue}{Client} \textcolor{blue}{Service} \textcolor{blue}{Back} \textcolor{blue}{Office} as part of the Small Claims Registry Operations \textcolor{green}{receives} \textcolor{orange}{the} \textcolor{orange}{request} and \textcolor{green}{retrieves} \textcolor{orange}{the} \textcolor{orange}{SCT} \textcolor{orange}{file}. Then, \textcolor{orange}{the} \textcolor{orange}{SCT} \textcolor{orange}{Warrant} \textcolor{orange}{Possession} is \textcolor{green}{forwarded} to \textcolor{blue}{Queensland} \textcolor{blue}{Police}. \textcolor{orange}{The} \textcolor{orange}{SCT} \textcolor{orange}{physical} \textcolor{orange}{file} is \textcolor{green}{stored} by \textcolor{blue}{the} \textcolor{blue}{Back} \textcolor{blue}{Office} \textcolor{green}{awaiting} \textcolor{orange}{a} \textcolor{orange}{report} \textcolor{purple}{to} \textcolor{purple}{be} \textcolor{purple}{sent} \textcolor{purple}{by} \textcolor{purple}{the} \textcolor{purple}{Police}. When \textcolor{orange}{the} \textcolor{orange}{report} is \textcolor{green}{received}, \textcolor{orange}{the} \textcolor{orange}{respective} \textcolor{orange}{SCT} \textcolor{orange}{file} is \textcolor{green}{retrieved}. Then, \textcolor{blue}{Back} \textcolor{blue}{Office} \textcolor{green}{attaches} \textcolor{orange}{the} \textcolor{orange}{new} \textcolor{orange}{SCT} \textcolor{orange}{document}, and \textcolor{green}{stores} \textcolor{orange}{the} \textcolor{orange}{expanded} \textcolor{orange}{SCT} \textcolor{orange}{physical} \textcolor{orange}{file}. After that, \textcolor{blue}{some} \textcolor{blue}{other} \textcolor{blue}{MC} \textcolor{blue}{internal} \textcolor{blue}{staff} \textcolor{green}{receives} \textcolor{orange}{the} \textcolor{orange}{physical} \textcolor{orange}{SCT} \textcolor{orange}{file} (out of scope).

\textbf{Text 5: Process Description 3-5: 2009-5 P\&E - Lodge Originating Document by Post}\\
\textcolor{orange}{Mail} \textcolor{orange}{from} \textcolor{orange}{the} \textcolor{orange}{party} is \textcolor{green}{collected} on a daily basis by \textcolor{blue}{the} \textcolor{blue}{Mail} \textcolor{blue}{Processing} \textcolor{blue}{Unit}. Within this unit, \textcolor{blue}{the} \textcolor{blue}{Mail} \textcolor{blue}{Clerk} \textcolor{green}{sorts} \textcolor{orange}{the} \textcolor{orange}{unopened} \textcolor{orange}{mail} \textcolor{purple}{into} \textcolor{purple}{the} \textcolor{purple}{various} \textcolor{purple}{business} \textcolor{purple}{areas}. \textcolor{orange}{The} \textcolor{orange}{mail} is then \textcolor{green}{distributed}. When \textcolor{orange}{the} \textcolor{orange}{mail} is \textcolor{green}{received} by \textcolor{blue}{the} \textcolor{blue}{Registry}, \textcolor{orange}{it} is \textcolor{green}{opened} and \textcolor{green}{sorted} into groups for distribution, and thus \textcolor{green}{registered} \textcolor{purple}{in} \textcolor{purple}{a} \textcolor{purple}{manual} \textcolor{purple}{incoming} \textcolor{purple}{Mail} \textcolor{purple}{Register}. Afterwards, \textcolor{blue}{the} \textcolor{blue}{Assistant} \textcolor{blue}{Registry} \textcolor{blue}{Manager} within the Registry \textcolor{green}{performs} \textcolor{green}{a} \textcolor{green}{quality} \textcolor{green}{check}. \textcolor{red}{If} \textcolor{brown}{the} \textcolor{brown}{mail} \textcolor{brown}{is} \textcolor{brown}{not} \textcolor{brown}{compliant}, \textcolor{orange}{a} \textcolor{orange}{list} \textcolor{orange}{of} \textcolor{orange}{requisition} explaining the reason for rejection is \textcolor{green}{compiled} and \textcolor{green}{sent} \textcolor{green}{back} to \textcolor{blue}{the} \textcolor{blue}{party}. \textcolor{red}{Otherwise}, \textcolor{orange}{the} \textcolor{orange}{matter} \textcolor{orange}{details} (types of action) are \textcolor{green}{captured} and \textcolor{green}{provided} to \textcolor{blue}{the} \textcolor{blue}{Cashier}, \textcolor{blue}{who} \textcolor{green}{takes} the applicable fees attached to the mail. At this point, \textcolor{blue}{the} \textcolor{blue}{Assistant} \textcolor{blue}{Registry} \textcolor{blue}{Manager} \textcolor{green}{puts} \textcolor{orange}{the} \textcolor{orange}{receipt} \textcolor{orange}{and} \textcolor{orange}{copied} \textcolor{orange}{documents} \textcolor{purple}{into} \textcolor{purple}{an} \textcolor{purple}{envelope} and \textcolor{green}{posts} \textcolor{orange}{it} to \textcolor{blue}{the} \textcolor{blue}{party}. \textcolor{red}{Meantime}, \textcolor{blue}{the} \textcolor{blue}{Cashier} \textcolor{green}{captures} \textcolor{orange}{the} \textcolor{orange}{Party} \textcolor{orange}{Details} and \textcolor{green}{prints} \textcolor{orange}{the} \textcolor{orange}{Physical} \textcolor{orange}{Court} \textcolor{orange}{File}.

\textbf{Text 6: Process Description 3-6: 2010-1 Claims Notification}\\
When \textcolor{orange}{a} \textcolor{orange}{claim} is \textcolor{green}{received}, \textcolor{orange}{it} is first \textcolor{green}{checked} whether the claimant is insured by the organization. \textcolor{red}{If} \textcolor{brown}{not}, \textcolor{blue}{the} \textcolor{blue}{claimant} is \textcolor{green}{informed} that \textcolor{orange}{the} \textcolor{orange}{claim} \textcolor{orange}{must} \textcolor{orange}{be} \textcolor{orange}{rejected}. \textcolor{red}{Otherwise}, \textcolor{orange}{the} \textcolor{orange}{severity} \textcolor{orange}{of} \textcolor{orange}{the} \textcolor{orange}{claim} is \textcolor{green}{evaluated}. Based on the outcome (simple or complex claims), \textcolor{orange}{relevant} \textcolor{orange}{forms} are \textcolor{green}{sent} to \textcolor{blue}{the} \textcolor{blue}{claimant}. Once \textcolor{orange}{the} \textcolor{orange}{forms} are \textcolor{green}{returned}, \textcolor{orange}{they} are \textcolor{green}{checked} \textcolor{purple}{for} \textcolor{purple}{completeness}. \textcolor{red}{If} \textcolor{brown}{the} \textcolor{brown}{forms} \textcolor{brown}{provide} \textcolor{brown}{all} \textcolor{brown}{relevant} \textcolor{brown}{details}, \textcolor{orange}{the} \textcolor{orange}{claim} is \textcolor{green}{registered} \textcolor{purple}{in} \textcolor{purple}{the} \textcolor{purple}{Claims} \textcolor{purple}{Management} \textcolor{purple}{system}, which ends the Claims Notification process. \textcolor{red}{Otherwise}, \textcolor{blue}{the} \textcolor{blue}{claimant} is \textcolor{green}{informed} \textcolor{purple}{to} \textcolor{purple}{update} \textcolor{purple}{the} \textcolor{purple}{forms}. Upon reception of the updated forms, they are checked again.

\textbf{Text 7: Process Description 3-8: 2010-3 Claims Handling Process}\\
The process starts when \textcolor{blue}{a} \textcolor{blue}{customer} \textcolor{green}{submits} \textcolor{orange}{a} \textcolor{orange}{claim} \textcolor{purple}{by} \textcolor{purple}{sending} \textcolor{purple}{in} \textcolor{purple}{relevant} \textcolor{purple}{documentation}. \textcolor{blue}{The} \textcolor{blue}{Notification} \textcolor{blue}{department} at the car insurer \textcolor{green}{checks} \textcolor{orange}{the} \textcolor{orange}{documents} \textcolor{purple}{upon} \textcolor{purple}{completeness} and \textcolor{green}{registers} \textcolor{orange}{the} \textcolor{orange}{claim}. Then, \textcolor{blue}{the} \textcolor{blue}{Handling} \textcolor{blue}{department} \textcolor{green}{picks} \textcolor{green}{up} \textcolor{orange}{the} \textcolor{orange}{claim} and \textcolor{green}{checks} \textcolor{orange}{the} \textcolor{orange}{insurance}. Then, \textcolor{orange}{an} \textcolor{orange}{assessment} is \textcolor{green}{performed}. \textcolor{red}{If} \textcolor{brown}{the} \textcolor{brown}{assessment} \textcolor{brown}{is} \textcolor{brown}{positive}, \textcolor{blue}{a} \textcolor{blue}{garage} is \textcolor{green}{phoned} \textcolor{purple}{to} \textcolor{purple}{authorise} \textcolor{purple}{the} \textcolor{purple}{repairs} and \textcolor{orange}{the} \textcolor{orange}{payment} is \textcolor{green}{scheduled} (in this order). \textcolor{red}{Otherwise}, \textcolor{orange}{the} \textcolor{orange}{claim} is \textcolor{green}{rejected}. In any case (whether the outcome is positive or negative), \textcolor{orange}{a} \textcolor{orange}{letter} is \textcolor{green}{sent} to \textcolor{blue}{the} \textcolor{blue}{customer} and the process is considered to be complete.

\textbf{Text 8: Process Description 5-1: Active VOS Tutorial}\\
The loan approval process starts by \textcolor{green}{receiving} a customer request for a loan amount. \textcolor{blue}{The} \textcolor{blue}{risk} \textcolor{blue}{assessment} \textcolor{blue}{Web} \textcolor{blue}{service} is invoked to \textcolor{green}{assess} \textcolor{orange}{the} \textcolor{orange}{request}. \textcolor{red}{If} \textcolor{brown}{the} \textcolor{brown}{loan} \textcolor{brown}{is} \textcolor{brown}{small} \textcolor{brown}{and} \textcolor{brown}{the} \textcolor{brown}{customer} \textcolor{brown}{is} \textcolor{brown}{low} \textcolor{brown}{risk}, \textcolor{orange}{the} \textcolor{orange}{loan} is \textcolor{green}{approved}. \textcolor{red}{If} \textcolor{brown}{the} \textcolor{brown}{customer} \textcolor{brown}{is} \textcolor{brown}{high} \textcolor{brown}{risk}, \textcolor{orange}{the} \textcolor{orange}{loan} is \textcolor{green}{denied}. \textcolor{red}{If} \textcolor{brown}{the} \textcolor{brown}{customer} \textcolor{brown}{needs} \textcolor{brown}{further} \textcolor{brown}{review} \textcolor{brown}{or} \textcolor{brown}{the} \textcolor{brown}{loan} \textcolor{brown}{amount} \textcolor{brown}{is} \textcolor{brown}{for} \textcolor{brown}{\$10,000} \textcolor{brown}{or} \textcolor{brown}{more}, \textcolor{orange}{the} \textcolor{orange}{request} is \textcolor{green}{sent} to \textcolor{blue}{the} \textcolor{blue}{approver} \textcolor{blue}{Web} \textcolor{blue}{service}. \textcolor{blue}{The} \textcolor{blue}{customer} \textcolor{green}{receives} \textcolor{orange}{feedback} from \textcolor{blue}{the} \textcolor{blue}{assessor} \textcolor{blue}{or} \textcolor{blue}{approver}.

\textbf{Text 9: Process Description 5-2: BizAgi Tutorial 1}\\
The process of Vacations Request starts when \textcolor{blue}{any} \textcolor{blue}{employee} of the organization \textcolor{green}{submits} \textcolor{orange}{a} \textcolor{orange}{vacation} \textcolor{orange}{request}. Once \textcolor{orange}{the} \textcolor{orange}{requirement} is \textcolor{green}{registered}, \textcolor{orange}{the} \textcolor{orange}{request} is \textcolor{green}{received} by \textcolor{blue}{the} \textcolor{blue}{immediate} \textcolor{blue}{supervisor} of the employee requesting the vacation. \textcolor{blue}{The} \textcolor{blue}{supervisor} must \textcolor{green}{approve} \textcolor{green}{or} \textcolor{green}{reject} \textcolor{orange}{the} \textcolor{orange}{request}. \textcolor{red}{If} \textcolor{brown}{the} \textcolor{brown}{request} \textcolor{brown}{is} \textcolor{brown}{rejected}, \textcolor{orange}{the} \textcolor{orange}{application} is \textcolor{green}{returned} to \textcolor{blue}{the} \textcolor{blue}{applicant} \textcolor{blue}{/} \textcolor{blue}{employee} who can review the rejection reasons. \textcolor{red}{If} \textcolor{brown}{the} \textcolor{brown}{request} \textcolor{brown}{is} \textcolor{brown}{approved} \textcolor{orange}{a} \textcolor{orange}{notification} is \textcolor{green}{generated} to \textcolor{blue}{the} \textcolor{blue}{Human} \textcolor{blue}{Resources} \textcolor{blue}{Representative}, \textcolor{blue}{who} must \textcolor{green}{complete} \textcolor{orange}{the} \textcolor{orange}{respective} \textcolor{orange}{management} \textcolor{orange}{procedures}.

\textbf{Text 10: Process Description 5-3: BizAgi Tutorial 2}\\
The process of an Office Supply Request starts when \textcolor{blue}{any} \textcolor{blue}{employee} of the organization \textcolor{green}{submits} \textcolor{orange}{an} \textcolor{orange}{office} \textcolor{orange}{supply} \textcolor{orange}{request}. Once \textcolor{orange}{the} \textcolor{orange}{requirement} is \textcolor{green}{registered}, \textcolor{orange}{the} \textcolor{orange}{request} is \textcolor{green}{received} by \textcolor{blue}{the} \textcolor{blue}{immediate} \textcolor{blue}{supervisor} \textcolor{blue}{of} \textcolor{blue}{the} \textcolor{blue}{employee} requesting the office supplies. \textcolor{blue}{The} \textcolor{blue}{supervisor} must \textcolor{green}{approve} \textcolor{green}{or} \textcolor{green}{ask} \textcolor{green}{for} \textcolor{green}{a} \textcolor{green}{change}\textcolor{green}{,} \textcolor{green}{or} \textcolor{green}{reject} \textcolor{orange}{the} \textcolor{orange}{request}. \textcolor{red}{If} \textcolor{brown}{the} \textcolor{brown}{request} \textcolor{brown}{is} \textcolor{brown}{rejected} the process will end. \textcolor{red}{If} \textcolor{brown}{the} \textcolor{brown}{request} \textcolor{brown}{is} \textcolor{brown}{asked} \textcolor{brown}{to} \textcolor{brown}{make} \textcolor{brown}{a} \textcolor{brown}{change} then \textcolor{orange}{it} is \textcolor{green}{returned} to \textcolor{blue}{the} \textcolor{blue}{petitioner} \textcolor{blue}{/} \textcolor{blue}{employee} who can \textcolor{green}{review} \textcolor{orange}{the} \textcolor{orange}{comments} \textcolor{orange}{for} \textcolor{orange}{the} \textcolor{orange}{change} \textcolor{orange}{request}. \textcolor{red}{If} \textcolor{brown}{the} \textcolor{brown}{request} \textcolor{brown}{is} \textcolor{brown}{approved} \textcolor{orange}{it} will \textcolor{green}{go} to \textcolor{blue}{the} \textcolor{blue}{purchase} \textcolor{blue}{department} that will be in charge of \textcolor{green}{making} \textcolor{orange}{quotations} (using a subprocess) and \textcolor{green}{select} \textcolor{orange}{a} \textcolor{orange}{vendor}. \textcolor{red}{If} \textcolor{brown}{the} \textcolor{brown}{vendor} \textcolor{brown}{is} \textcolor{brown}{not} \textcolor{brown}{valid} \textcolor{brown}{in} \textcolor{brown}{the} \textcolor{brown}{system} \textcolor{blue}{the} \textcolor{blue}{purchase} \textcolor{blue}{department} will have to \textcolor{green}{choose} \textcolor{orange}{a} \textcolor{orange}{different} \textcolor{orange}{vendor}. After \textcolor{orange}{a} \textcolor{orange}{vendor} is \textcolor{green}{selected} and \textcolor{green}{confirmed}, \textcolor{blue}{the} \textcolor{blue}{system} will \textcolor{green}{generate} and \textcolor{green}{send} \textcolor{orange}{a} \textcolor{orange}{purchase} \textcolor{orange}{order} and \textcolor{green}{wait} \textcolor{green}{for} \textcolor{orange}{the} \textcolor{orange}{product} \textcolor{orange}{to} \textcolor{orange}{be} \textcolor{orange}{delivered} and \textcolor{orange}{the} \textcolor{orange}{invoice} \textcolor{orange}{received}. In any case the system will send a notification to let the user know what the result was. In any of the cases, approval, rejection or change required \textcolor{blue}{the} \textcolor{blue}{system} will \textcolor{green}{send} \textcolor{blue}{the} \textcolor{blue}{user} \textcolor{orange}{a} \textcolor{orange}{notification}.

\textbf{Text 11: Process Description 8-2: HR Process - HR Department}\\
I am the HR clerk. When \textcolor{orange}{a} \textcolor{orange}{vacancy} is \textcolor{green}{reported} to \textcolor{blue}{me}, \textcolor{blue}{I} \textcolor{green}{create} \textcolor{orange}{a} \textcolor{orange}{job} \textcolor{orange}{description} \textcolor{orange}{from} \textcolor{orange}{the} \textcolor{orange}{information}. \textcolor{red}{Sometimes} \textcolor{brown}{there} \textcolor{brown}{is} \textcolor{brown}{still} \textcolor{brown}{confusion} \textcolor{brown}{in} \textcolor{brown}{the} \textcolor{brown}{message}, then \textcolor{blue}{I} must \textcolor{green}{ask} \textcolor{blue}{the} \textcolor{blue}{Department} again. \textcolor{blue}{I} am \textcolor{green}{submitting} \textcolor{orange}{the} \textcolor{orange}{job} \textcolor{orange}{description} for consideration and \textcolor{green}{waiting} for \textcolor{orange}{the} \textcolor{orange}{approval}. But, \textcolor{red}{it} \textcolor{red}{can} \textcolor{red}{also} \textcolor{red}{happen} \textcolor{red}{that} \textcolor{brown}{the} \textcolor{brown}{department} \textcolor{brown}{does} \textcolor{brown}{not} \textcolor{brown}{approve} \textcolor{brown}{it}, but rejects it, and \textcolor{green}{requests} \textcolor{orange}{a} \textcolor{orange}{correction}. Then \textcolor{blue}{I} \textcolor{green}{correct} \textcolor{orange}{the} \textcolor{orange}{description} and \textcolor{green}{submit} \textcolor{orange}{it} again for consideration. \textcolor{red}{If} \textcolor{brown}{the} \textcolor{brown}{description} \textcolor{brown}{is} \textcolor{brown}{finally} \textcolor{brown}{approved}, \textcolor{blue}{I} \textcolor{green}{post} \textcolor{orange}{the} \textcolor{orange}{job}.

\textbf{Text 12: Process Description 6-3}\\
Every time we \textcolor{green}{get} \textcolor{orange}{a} \textcolor{orange}{new} \textcolor{orange}{order} from \textcolor{blue}{the} \textcolor{blue}{sales} \textcolor{blue}{department}, first, \textcolor{blue}{one} \textcolor{blue}{of} \textcolor{blue}{my} \textcolor{blue}{masters} \textcolor{green}{determines} \textcolor{orange}{the} \textcolor{orange}{necessary} \textcolor{orange}{parts} \textcolor{orange}{and} \textcolor{orange}{quantities} \textcolor{orange}{as} \textcolor{orange}{well} \textcolor{orange}{as} \textcolor{orange}{the} \textcolor{orange}{delivery} \textcolor{orange}{date}. Once that information is present, \textcolor{orange}{it} has to be \textcolor{green}{entered} into our production planning system (PPS). It optimizes our production processes and creates possibly uniform work packages so that the setup times are minimized. Besides, \textcolor{blue}{it} \textcolor{green}{creates} \textcolor{orange}{a} \textcolor{orange}{list} \textcolor{orange}{of} \textcolor{orange}{parts} to be procured. Unfortunately it is not coupled correctly to our Enterprise Resource Planning system (ERP), so \textcolor{orange}{the} \textcolor{orange}{data} must be \textcolor{green}{transferred} \textcolor{purple}{manually}. By the way, that is the second step. Once all the data is present, we need to decide whether \textcolor{brown}{any} \textcolor{brown}{parts} \textcolor{brown}{are} \textcolor{brown}{missing} and must be \textcolor{green}{procured} \textcolor{red}{or} if this is not necessary. Once production is scheduled to start, \textcolor{blue}{we} \textcolor{green}{receive} \textcolor{orange}{a} \textcolor{orange}{notice} from the system and \textcolor{blue}{an} \textcolor{blue}{employee} \textcolor{green}{takes} \textcolor{green}{care} of \textcolor{orange}{the} \textcolor{orange}{implementation}. Finally, \textcolor{orange}{the} \textcolor{orange}{order} will be \textcolor{green}{checked} again \textcolor{purple}{for} \textcolor{purple}{its} \textcolor{purple}{quality}.

\textbf{Text 13: Process Description 6-4: Turbopixel}\\
The first step is to \textcolor{green}{determine} \textcolor{orange}{contact} \textcolor{orange}{details} of potential customers. This can be achieved in several ways. \textcolor{red}{Sometimes}, \textcolor{blue}{we} \textcolor{green}{buy} \textcolor{orange}{details} \textcolor{orange}{for} \textcolor{orange}{cold} \textcolor{orange}{calls}, \textcolor{red}{sometimes}, \textcolor{blue}{our} \textcolor{blue}{marketing} \textcolor{blue}{staff} \textcolor{green}{participates} in exhibitions and \textcolor{red}{sometimes}, \textcolor{blue}{you} just happen to \textcolor{green}{know} \textcolor{orange}{somebody}\textcolor{orange}{,} \textcolor{orange}{who} \textcolor{orange}{is} \textcolor{orange}{interested} \textcolor{orange}{in} \textcolor{orange}{the} \textcolor{orange}{product}. Then \textcolor{blue}{we} start \textcolor{green}{calling} \textcolor{blue}{the} \textcolor{blue}{customer}. That is done by the call center staff. \textcolor{blue}{They} are \textcolor{green}{determining} \textcolor{orange}{the} \textcolor{orange}{contact} \textcolor{orange}{person} and \textcolor{orange}{the} \textcolor{orange}{budget} which would be available for the project. Of course, \textcolor{green}{asking} \textcolor{blue}{the} \textcolor{blue}{customer} whether he is generally interested is also important. \textcolor{red}{If} \textcolor{brown}{this} \textcolor{brown}{is} \textcolor{brown}{not} \textcolor{brown}{the} \textcolor{brown}{case}, we \textcolor{green}{leave} \textcolor{orange}{him} alone, except \textcolor{red}{if} \textcolor{brown}{the} \textcolor{brown}{potential} \textcolor{brown}{project} \textcolor{brown}{budget} \textcolor{brown}{is} \textcolor{brown}{huge}. Then \textcolor{blue}{the} \textcolor{blue}{head} \textcolor{blue}{of} \textcolor{blue}{development} personally tries to \textcolor{green}{acquire} \textcolor{orange}{the} \textcolor{orange}{customer}. \textcolor{red}{If} \textcolor{brown}{the} \textcolor{brown}{customer} \textcolor{brown}{is} \textcolor{brown}{interested} \textcolor{brown}{in} \textcolor{brown}{the} \textcolor{brown}{end}, the next step is \textcolor{orange}{a} \textcolor{orange}{detailed} \textcolor{orange}{online} \textcolor{orange}{presentation}. It is \textcolor{green}{given} either by a sales representative or by a pre-sales employee in case of a more technical presentation. Afterwards we are \textcolor{green}{waiting} \textcolor{green}{for} \textcolor{orange}{the} \textcolor{orange}{customer} \textcolor{orange}{to} \textcolor{orange}{come} \textcolor{orange}{back} \textcolor{orange}{to} \textcolor{orange}{us}. \textcolor{red}{If} \textcolor{brown}{we} \textcolor{brown}{are} \textcolor{brown}{not} \textcolor{brown}{contacted} \textcolor{brown}{within} \textcolor{brown}{2} \textcolor{brown}{weeks}, \textcolor{blue}{a} \textcolor{blue}{sales} \textcolor{blue}{representative} is \textcolor{green}{calling} \textcolor{blue}{the} \textcolor{blue}{customer}. The last phase is the \textcolor{green}{creation} of \textcolor{orange}{a} \textcolor{orange}{quotation}.

\textbf{Text 14: Process Description 6-1: ACME}\\
As a basic principle, \textcolor{blue}{ACME} \textcolor{blue}{AG} \textcolor{green}{receives} \textcolor{orange}{invoices} on paper or fax. \textcolor{orange}{These} are \textcolor{green}{received} by \textcolor{blue}{the} \textcolor{blue}{Secretariat} in the central inbox and \textcolor{green}{forwarded} after \textcolor{purple}{a} \textcolor{purple}{short} \textcolor{purple}{visual} \textcolor{green}{inspection} to \textcolor{blue}{an} \textcolor{blue}{accounting} \textcolor{blue}{employee}. In "ACME Financial Accounting", a software specially developed for the ACME AG, \textcolor{blue}{she} \textcolor{green}{identifies} \textcolor{orange}{the} \textcolor{orange}{charging} \textcolor{orange}{suppliers} and \textcolor{green}{creates} \textcolor{orange}{a} \textcolor{orange}{new} \textcolor{orange}{instance} (invoice). \textcolor{blue}{She} then \textcolor{green}{checks} \textcolor{orange}{the} \textcolor{orange}{invoice} \textcolor{orange}{items} and \textcolor{green}{notes} \textcolor{orange}{the} \textcolor{orange}{corresponding} \textcolor{orange}{cost} \textcolor{orange}{center} \textcolor{orange}{at} \textcolor{orange}{the} \textcolor{orange}{ACME} \textcolor{orange}{AG} \textcolor{orange}{and} \textcolor{orange}{the} \textcolor{orange}{related} \textcolor{orange}{cost} \textcolor{orange}{center} \textcolor{orange}{managers} \textcolor{orange}{for} \textcolor{orange}{each} \textcolor{orange}{position} on a separate form ("docket"). \textcolor{orange}{The} \textcolor{orange}{docket} \textcolor{orange}{and} \textcolor{orange}{the} \textcolor{orange}{copy} \textcolor{orange}{of} \textcolor{orange}{the} \textcolor{orange}{invoice} \textcolor{green}{go} to the internal mail together and are \textcolor{green}{sent} to \textcolor{blue}{the} \textcolor{blue}{first} \textcolor{blue}{cost} \textcolor{blue}{center} \textcolor{blue}{manager} to the list. \textcolor{blue}{He} \textcolor{green}{reviews} \textcolor{orange}{the} \textcolor{orange}{content} \textcolor{purple}{for} \textcolor{purple}{accuracy} after \textcolor{green}{receiving} \textcolor{orange}{the} \textcolor{orange}{copy} \textcolor{orange}{of} \textcolor{orange}{the} \textcolor{orange}{invoice}. \textcolor{red}{Should} \textcolor{brown}{everything} \textcolor{brown}{be} \textcolor{brown}{in} \textcolor{brown}{order}, \textcolor{blue}{he} \textcolor{green}{notes} \textcolor{orange}{his} \textcolor{orange}{code} one \textcolor{purple}{on} \textcolor{purple}{the} \textcolor{purple}{docket} ("accurate position - AP") and \textcolor{green}{returns} \textcolor{orange}{the} \textcolor{orange}{copy} \textcolor{orange}{of} \textcolor{orange}{the} \textcolor{orange}{invoice} to \textcolor{blue}{the} \textcolor{blue}{internal} \textcolor{blue}{mail}. From it, \textcolor{orange}{the} \textcolor{orange}{copy} \textcolor{orange}{of} \textcolor{orange}{the} \textcolor{orange}{invoice} is \textcolor{green}{passed} \textcolor{green}{on} to \textcolor{blue}{the} \textcolor{blue}{next} \textcolor{blue}{cost} \textcolor{blue}{center} \textcolor{blue}{manager}, based on the docket, or \textcolor{red}{if} \textcolor{brown}{all} \textcolor{brown}{items} \textcolor{brown}{are} \textcolor{brown}{marked} \textcolor{brown}{correct}, \textcolor{green}{sent} \textcolor{green}{back} to \textcolor{blue}{accounting}. Therefore, the copy of invoice and the docket gradually move through the hands of all cost center managers until all positions are marked as completely accurate. However, \textcolor{red}{if} \textcolor{brown}{inconsistencies} \textcolor{brown}{exist}, e.g. because the ordered product is not of the expected quantity or quality, \textcolor{blue}{the} \textcolor{blue}{cost} \textcolor{blue}{center} \textcolor{blue}{manager} \textcolor{green}{rejects} \textcolor{orange}{the} \textcolor{orange}{AP} with a note and explanatory statement on the docket, and \textcolor{orange}{the} \textcolor{orange}{copy} \textcolor{orange}{of} \textcolor{orange}{the} \textcolor{orange}{invoice} is \textcolor{green}{sent} \textcolor{green}{back} to \textcolor{blue}{accounting} directly. Based on the statements of the cost center managers, \textcolor{blue}{she} will \textcolor{green}{proceede} with \textcolor{orange}{the} \textcolor{orange}{clarification} \textcolor{orange}{with} \textcolor{orange}{the} \textcolor{orange}{vendor}, but, \textcolor{red}{if} \textcolor{brown}{necessary}, \textcolor{blue}{she} \textcolor{green}{consults} \textcolor{blue}{the} \textcolor{blue}{cost} \textcolor{blue}{center} \textcolor{blue}{managers} by telephone or e-mail again. When all inconsistencies are resolved, \textcolor{orange}{the} \textcolor{orange}{copy} \textcolor{orange}{of} \textcolor{orange}{the} \textcolor{orange}{invoice} is \textcolor{green}{sent} to \textcolor{blue}{the} \textcolor{blue}{cost} \textcolor{blue}{center} \textcolor{blue}{managers} again, and the process continues. After all invoice items are AP, \textcolor{blue}{the} \textcolor{blue}{accounting} \textcolor{blue}{employee} \textcolor{green}{forwards} \textcolor{orange}{the} \textcolor{orange}{copy} \textcolor{orange}{of} \textcolor{orange}{the} \textcolor{orange}{invoice} to \textcolor{blue}{the} \textcolor{blue}{commercial} \textcolor{blue}{manager}. \textcolor{blue}{He} \textcolor{green}{makes} \textcolor{orange}{the} \textcolor{orange}{commercial} \textcolor{orange}{audit} and \textcolor{green}{issues} \textcolor{orange}{the} \textcolor{orange}{approval} \textcolor{orange}{for} \textcolor{orange}{payment}. \textcolor{red}{If} \textcolor{brown}{the} \textcolor{brown}{bill} \textcolor{brown}{amount} \textcolor{brown}{exceeds} \textcolor{brown}{EUR} \textcolor{brown}{20}, \textcolor{blue}{the} \textcolor{blue}{Board} wants to \textcolor{green}{check} \textcolor{orange}{it} again (4 - eyes-principle). \textcolor{orange}{The} \textcolor{orange}{copy} \textcolor{orange}{of} \textcolor{orange}{the} \textcolor{orange}{invoice} \textcolor{orange}{including} \textcolor{orange}{the} \textcolor{orange}{docket} \textcolor{green}{moves} \textcolor{green}{back} to \textcolor{blue}{the} \textcolor{blue}{accounting} \textcolor{blue}{employee} \textcolor{purple}{in} \textcolor{purple}{the} \textcolor{purple}{appropriate} \textcolor{purple}{signature} \textcolor{purple}{file}. \textcolor{red}{Should} \textcolor{brown}{there} \textcolor{brown}{be} \textcolor{brown}{a} \textcolor{brown}{complaint} \textcolor{brown}{during} \textcolor{brown}{the} \textcolor{brown}{commercial} \textcolor{brown}{audit}, \textcolor{orange}{it} will be \textcolor{green}{resolved} by \textcolor{blue}{the} \textcolor{blue}{accounting} \textcolor{blue}{employee} with the supplier. After the commercial audit is successfully completed, \textcolor{blue}{the} \textcolor{blue}{accounting} \textcolor{blue}{employee} \textcolor{green}{gives} \textcolor{orange}{payment} \textcolor{orange}{instructions} and \textcolor{green}{closes} \textcolor{orange}{the} \textcolor{orange}{instance} \textcolor{orange}{in} \textcolor{orange}{"} \textcolor{orange}{ACME} \textcolor{orange}{financial} \textcolor{orange}{accounting} \textcolor{orange}{"}.

\textbf{Text 15: Process Description 2-2: Supplier Switch}\\
The process is initiated by a switch-over request. In doing so, \textcolor{blue}{the} \textcolor{blue}{customer} \textcolor{green}{transmits} \textcolor{orange}{his} \textcolor{orange}{data} to \textcolor{blue}{the} \textcolor{blue}{customer} \textcolor{blue}{service} \textcolor{blue}{department} of the company. Customer service is a shared service center between the departments Sales and Distribution. \textcolor{orange}{The} \textcolor{orange}{customer} \textcolor{orange}{data} is \textcolor{green}{received} by \textcolor{blue}{customer} \textcolor{blue}{service} and based on this data \textcolor{orange}{a} \textcolor{orange}{customer} \textcolor{orange}{data} \textcolor{orange}{object} is \textcolor{green}{entered} \textcolor{purple}{into} \textcolor{purple}{the} \textcolor{purple}{CRM} \textcolor{purple}{system}. After customer data has been entered \textcolor{orange}{it} should then be \textcolor{green}{compared} with \textcolor{orange}{the} \textcolor{orange}{internal} \textcolor{orange}{customer} \textcolor{orange}{data} \textcolor{orange}{base} and \textcolor{green}{checked} \textcolor{purple}{for} \textcolor{purple}{completeness} \textcolor{purple}{and} \textcolor{purple}{plausibility}. \textcolor{red}{In} \textcolor{red}{case} \textcolor{red}{of} \textcolor{brown}{any} \textcolor{brown}{errors} \textcolor{orange}{these} should be \textcolor{green}{corrected} on the basis of a simple error list. The comparison of data is done to prevent individual customer data being stored multiple times. \textcolor{red}{In} \textcolor{red}{case} \textcolor{brown}{the} \textcolor{brown}{customer} \textcolor{brown}{does} \textcolor{brown}{not} \textcolor{brown}{exist} \textcolor{brown}{in} \textcolor{brown}{the} \textcolor{brown}{customer} \textcolor{brown}{data} \textcolor{brown}{base}, \textcolor{orange}{a} \textcolor{orange}{new} \textcolor{orange}{customer} \textcolor{orange}{object} is being \textcolor{green}{created} which will remain the data object of interest during the rest of the process flow. This object consists of data elements such as the customer's name and address and the assigned power gauge. The generated customer object is then used, in combination with other customer data to \textcolor{green}{prepare} \textcolor{orange}{the} \textcolor{orange}{contract} \textcolor{orange}{documents} \textcolor{orange}{for} \textcolor{orange}{the} \textcolor{orange}{power} \textcolor{orange}{supplier} \textcolor{orange}{switch} (including data such as bank connection, information on the selected rate, requested date of switch-over). In the following an automated \textcolor{green}{check} of \textcolor{orange}{the} \textcolor{orange}{contract} \textcolor{orange}{documents} is carried out within \textcolor{blue}{the} \textcolor{blue}{CIS} (customer information system) in order to confirm their successful generation. \textcolor{red}{In} \textcolor{red}{case} \textcolor{red}{of} \textcolor{brown}{a} \textcolor{brown}{negative} \textcolor{brown}{response}, i.e. the contract documents are not (or incorrectly) generated, \textcolor{orange}{the} \textcolor{orange}{causing} \textcolor{orange}{issues} are being \textcolor{green}{analyzed} and \textcolor{green}{resolved}. Subsequently \textcolor{orange}{the} \textcolor{orange}{contract} \textcolor{orange}{documents} are \textcolor{green}{generated} once again. \textcolor{red}{In} \textcolor{red}{case} \textcolor{red}{of} \textcolor{brown}{a} \textcolor{brown}{positive} \textcolor{brown}{response} \textcolor{orange}{a} \textcolor{orange}{confirmation} \textcolor{orange}{document} is \textcolor{green}{sent} \textcolor{green}{out} to \textcolor{blue}{the} \textcolor{blue}{customer} stating that the switch-over to the new supplier can be executed. \textcolor{orange}{A} \textcolor{orange}{request} to \textcolor{blue}{the} \textcolor{blue}{grid} \textcolor{blue}{operator} is automatically \textcolor{green}{sent} \textcolor{green}{out} by \textcolor{blue}{the} \textcolor{blue}{CIS}. It puts the question whether the customer may be supplied by the selected supplier in the future. \textcolor{orange}{The} \textcolor{orange}{switch-over} \textcolor{orange}{request} is \textcolor{green}{checked} by \textcolor{blue}{the} \textcolor{blue}{grid} \textcolor{blue}{operator} for supplier concurrence, and \textcolor{blue}{the} \textcolor{blue}{grid} \textcolor{blue}{operator} \textcolor{green}{transmits} \textcolor{orange}{a} \textcolor{orange}{response} \textcolor{orange}{comment}. \textcolor{red}{In} \textcolor{red}{the} \textcolor{red}{case} \textcolor{red}{of} \textcolor{brown}{supplier} \textcolor{brown}{concurrence} \textcolor{blue}{the} \textcolor{blue}{grid} \textcolor{blue}{operator} would \textcolor{green}{inform} \textcolor{blue}{all} \textcolor{blue}{involved} \textcolor{blue}{suppliers} and \textcolor{green}{demand} \textcolor{orange}{the} \textcolor{orange}{resolution} \textcolor{orange}{of} \textcolor{orange}{the} \textcolor{orange}{conflict}. \textcolor{blue}{The} \textcolor{blue}{grid} \textcolor{blue}{operator} \textcolor{green}{communicates} with \textcolor{blue}{the} \textcolor{blue}{old} \textcolor{blue}{supplier} and \textcolor{green}{carries} \textcolor{green}{out} \textcolor{orange}{the} \textcolor{orange}{termination} \textcolor{orange}{of} \textcolor{orange}{the} \textcolor{orange}{sales} \textcolor{orange}{agreement} between the customer and the old supplier (i.e. the customer service (of the new supplier) does not have to interact with the old supplier regarding termination). \textcolor{red}{If} \textcolor{brown}{there} \textcolor{brown}{are} \textcolor{brown}{not} \textcolor{brown}{any} \textcolor{brown}{objections} \textcolor{brown}{by} \textcolor{brown}{the} \textcolor{brown}{grid} \textcolor{brown}{operator} (i.e. no supplier concurrence), \textcolor{blue}{customer} \textcolor{blue}{service} \textcolor{green}{creates} \textcolor{orange}{a} \textcolor{orange}{CIS} \textcolor{orange}{contract}. \textcolor{blue}{The} \textcolor{blue}{customer} then has the chance to \textcolor{green}{check} \textcolor{orange}{the} \textcolor{orange}{contract} \textcolor{orange}{details} and based on this check may decide to \textcolor{red}{either} \textcolor{green}{withdraw} from \textcolor{orange}{the} \textcolor{orange}{switch} \textcolor{orange}{contract} \textcolor{red}{or} \textcolor{green}{confirm} \textcolor{orange}{it}. Depending on the customer's acceptance / rejection the process flow at customer service either ends (in case of withdrawal) or continues (in case of a confirmation). An additional constraint is that the customer can only withdraw from the offered contract within 7 days after the 7th day the contract will be regarded as accepted and the process continues. The confirmation message by the customer is therefore not absolutely necessary (as it will count as accepted after 7 days in any way) but it can speed up the switch process. On the switch-date, but no later than 10 days after power supply has begun, \textcolor{blue}{the} \textcolor{blue}{grid} \textcolor{blue}{operator} \textcolor{green}{transmits} \textcolor{orange}{the} \textcolor{orange}{power} \textcolor{orange}{meter} \textcolor{orange}{data} to \textcolor{blue}{the} \textcolor{blue}{customer} \textcolor{blue}{service} and \textcolor{blue}{the} \textcolor{blue}{old} \textcolor{blue}{supplier} \textcolor{purple}{via} \textcolor{purple}{messages} \textcolor{purple}{containing} \textcolor{purple}{a} \textcolor{purple}{services} \textcolor{purple}{consumption} \textcolor{purple}{report}. \textcolor{red}{At} \textcolor{red}{the} \textcolor{red}{same} \textcolor{red}{time}, \textcolor{blue}{the} \textcolor{blue}{grid} \textcolor{blue}{operator} \textcolor{green}{computes} \textcolor{orange}{the} \textcolor{orange}{final} \textcolor{orange}{billing} based on the meter data and \textcolor{green}{sends} \textcolor{orange}{it} to \textcolor{blue}{the} \textcolor{blue}{old} \textcolor{blue}{supplier}. Likewise \textcolor{blue}{the} \textcolor{blue}{old} \textcolor{blue}{supplier} \textcolor{green}{creates} and \textcolor{green}{sends} \textcolor{orange}{the} \textcolor{orange}{final} \textcolor{orange}{billing} to \textcolor{blue}{the} \textcolor{blue}{customer}. For the customer as well as the grid operator the process ends then. After \textcolor{green}{receiving} \textcolor{orange}{the} \textcolor{orange}{meter} \textcolor{orange}{data} \textcolor{blue}{customer} \textcolor{blue}{service} \textcolor{green}{imports} \textcolor{orange}{the} \textcolor{orange}{meter} \textcolor{orange}{data} to systems that require the information. The process of winning a new customer ends here.


\textbf{Text 16: Process Description 0-1: BPIC 2020 Challenge\footnote{https://icpmconference.org/2020/bpi-challenge/}}\\
The various declaration documents (domestic and international declarations, pre-paid travel costs and requests for payment) all follow a similar process flow. After \textcolor{green}{submission} by \textcolor{blue}{the employee}, the request is \textcolor{green}{sent} \textcolor{orange}{for approval} to the travel administration. \textcolor{red}{If} \textcolor{brown}{approved}, \textcolor{orange}{the request} is then \textcolor{green}{forwarded} to \textcolor{blue}{the budget owner} and after that to \textcolor{blue}{the supervisor}. If the budget owner and supervisor are the same person, then only one of the these steps it taken. \textcolor{red}{In some cases}, \textcolor{blue}{the director} also needs to \textcolor{green}{approve} \textcolor{orange}{the request}. In all cases, a rejection leads to one of two outcomes. Either \textcolor{blue}{the employee} \textcolor{green}{resubmits} \textcolor{orange}{the request}, or \textcolor{blue}{the employee} also \textcolor{green}{rejects} \textcolor{orange}{the request}. \textcolor{red}{If} \textcolor{brown}{the approval flow has a positive result}, \textcolor{orange}{the payment} is \textcolor{green}{requested} and \textcolor{green}{made}.


\textbf{Text 17: Process Description 0-2: Customer Credit Request}\\
Whenever \textcolor{blue}{Elite Holdings} \textcolor{green}{receives} \textcolor{orange}{a customer request}, \textcolor{blue}{it} \textcolor{green}{demands} \textcolor{orange}{a solvency check} from Miracle Credit. At Miracle Credit exactly \textcolor{blue}{two clerks} \textcolor{green}{perform} \textcolor{orange}{a solvency check}. \textcolor{blue}{Miracle Credit} \textcolor{green}{hands back} the results of the solvency check to Elite Holding. \textcolor{red}{If} \textcolor{brown}{the solvency check is not passed}, \textcolor{blue}{a clerk} from the customer advisory \textcolor{green}{informs} \textcolor{orange}{the customer} and \textcolor{green}{deletes} \textcolor{orange}{the customer's request}. \textcolor{red}{If} \textcolor{brown}{the solvency check is passed}, \textcolor{blue}{Anna} or \textcolor{blue}{Hans}, bot not both, \textcolor{green}{develop} \textcolor{orange}{a payment schedule}. Afterward, the schedule is \textcolor{green}{sent} to the manager. Both \textcolor{blue}{he} and \textcolor{blue}{another clerk} with the role supervisor must \textcolor{green}{approve} \textcolor{orange}{the payment schedule}. Approve payment schedule may never be executed by Anna or Hans. \textcolor{red}{If} \textcolor{brown}{the payment schedule has been approved}, an email is \textcolor{green}{sent} to \textcolor{orange}{the customer} automatically, \textcolor{red}{otherwise}, \textcolor{blue}{the customer advisory} \textcolor{green}{calls} \textcolor{orange}{the customer} to suggest an alternative. In both cases, the request must be \textcolor{green}{closed}.


\textbf{Text 18: Process Description 0-3: Quince Harvesting \cite{text_01}}\\
The quince harvesting process takes place in October and November. Every day, from 7am \textcolor{blue}{the manager} \textcolor{green}{checks} that \textcolor{orange}{the plantation} has not been affected by codling moth. \textcolor{red}{If} \textcolor{brown}{it is affected} that day's \textcolor{orange}{production} is \textcolor{green}{interrupted}. \textcolor{blue}{The employees} begin to \textcolor{green}{pick} \textcolor{orange}{the fruits} at 8am, when the quinces have almost no dew. \textcolor{red}{If} \textcolor{brown}{the workers have not taken a break} before 1pm, \textcolor{blue}{the manager} \textcolor{green}{reminds} \textcolor{orange}{them} that they should take a break soon. 7 hours after the employees started picking the fruits, \textcolor{blue}{the trucks} \textcolor{green}{come} and \textcolor{blue}{the employees} \textcolor{green}{load} \textcolor{orange}{them} at most until 5pm. \textcolor{blue}{The supervisor} \textcolor{green}{notes down} \textcolor{orange}{the spoiled fruit} not later than 30 minutes after the trucks have been loaded. In the evening, \textcolor{blue}{the supervisor} \textcolor{green}{reports} \textcolor{orange}{the total number} of kilos collected and \textcolor{orange}{the hours} the employees have worked.


\textbf{Text 19: Process Description 0-4: Order-to-cash example \cite{text_02}}\\
The order-to-cash process is \textcolor{green}{carried out} by \textcolor{blue}{a seller’s organization} which includes two departments: the sales department and the warehouse \& distribution department. \textcolor{orange}{The purchase order} received by warehouse \& distribution is \textcolor{green}{checked} against the stock. This operation is carried out automatically by \textcolor{blue}{the ERP system} of warehouse \& distribution, which \textcolor{green}{queries} \textcolor{orange}{the warehouse database}. \textcolor{red}{If} \textcolor{brown}{the product is in stock}, \textcolor{orange}{it} is \textcolor{green}{retrieved} from the warehouse before the sales department confirms the order. Next, \textcolor{blue}{the sales department} \textcolor{green}{emits} \textcolor{orange}{an invoice} and \textcolor{green}{waits} for \textcolor{orange}{the payment}, while the product is \textcolor{green}{shipped} from within warehouse \& distribution. The process completes with the order archival in the sales department. \textcolor{red}{If} \textcolor{brown}{the product is not in stock}, \textcolor{blue}{the ERP system} within warehouse \& distribution \textcolor{green}{checks} \textcolor{orange}{the raw materials availability} by accessing the suppliers catalog. Once the raw materials have been obtained \textcolor{blue}{the warehouse \& distribution department} \textcolor{green}{takes care} of \textcolor{orange}{manufacturing the product}. The process completes with the purchase order being confirmed and archived by the sales department.


\textbf{Text 20: Process Description 0-5: reimbursing expenses \cite{text_02}}\\
After an expense report is received from \textcolor{blue}{an employee}, \textcolor{orange}{the employee} is \textcolor{green}{notified} of the receipt of the report. Next, \textcolor{orange}{a new account} must be \textcolor{green}{created} \textcolor{red}{if} \textcolor{brown}{the employee does not already have one}. The report is then \textcolor{green}{reviewed} for automatic approval. Amounts under e 1,000 are automatically \textcolor{green}{approved} while amounts equal to or over e 1,000 \textcolor{green}{require} manual approval. \textcolor{red}{In case of} \textcolor{brown}{rejection}, \textcolor{blue}{the employee} must \textcolor{green}{receive} \textcolor{orange}{a rejection notice} by email. \textcolor{red}{In case of} \textcolor{brown}{approval}, the reimbursement is \textcolor{green}{deposited} directly to the \textcolor{orange}{employee’s bank account} and \textcolor{orange}{an approval notice} is \textcolor{green}{sent} to the employee via email, with the details of the money transfer. At any time during the review, \textcolor{blue}{the employee} can \textcolor{green}{send} \textcolor{orange}{a request} for amount rectification. In that case the rectification is \textcolor{green}{registered} and \textcolor{orange}{the report} needs to be \textcolor{green}{reviewed} again. Moreover, if the report is not handled within 30 days, \textcolor{orange}{the process} is \textcolor{green}{stopped} and \textcolor{blue}{the employee} \textcolor{green}{receives} \textcolor{orange}{a cancelation notice email} so that \textcolor{blue}{he} can \textcolor{green}{resubmit} the \textcolor{orange}{expense report} from scratch.





\section{Dialogue of question answering with GPT}
\label{appendix:gpt}
\begin{itemize}
    \item \textbf{1-1:} https://chat.openai.com/share/cf6c568f-e7c6-4f74-a91c-55523cb73058
    \item \textbf{3-1:} https://chat.openai.com/share/06074abb-2f76-470c-9211-d75c59cdff40
    \item \textbf{3-5:} https://chat.openai.com/share/e1c47f83-1a80-42ce-ab22-5cad31ac0d2e
    \item \textbf{3-6:} https://chat.openai.com/share/4b1e9536-ca1e-407a-b283-44ec3e55027d
    \item \textbf{3-8:} https://chat.openai.com/share/c6801ee6-4241-4e6f-bbba-d1dfe18ce3aa
    \item \textbf{5-1:} https://chat.openai.com/share/37498138-65c4-4550-aeec-4894d0bed2dc
    \item \textbf{5-2:} https://chat.openai.com/share/13a10ca0-0ae8-4fc4-b3e0-642f12c91d51
    \item \textbf{5-3:} https://chat.openai.com/share/9bbe9145-dd08-4e46-a695-fcc09ed7c9c6
    \item \textbf{8-2:} https://chat.openai.com/share/98b4b935-ee13-4803-9fb1-78d857426845
    \item \textbf{6-3:} https://chat.openai.com/share/39171ee3-e1e2-458a-bb9e-160ad5f03255
    \item \textbf{6-4:} https://chat.openai.com/share/fe184dd2-22cb-4fef-9281-f7919cb21b31
    \item \textbf{6-1:} https://chat.openai.com/share/eadaef6d-50c2-4052-80d9-3feb64ae4629
    \item \textbf{2-2:} https://chat.openai.com/share/0dab3825-c26e-41a5-91ec-bf2a6bf964be
    \item \textbf{0-1:} https://chat.openai.com/share/d80a6de9-a356-4b30-a40e-20b9e84dfbdf
    \item \textbf{0-2:} https://chat.openai.com/share/fc2f2c62-a6e1-46e8-95e6-08c0684fecf1
    \item \textbf{0-3:} https://chat.openai.com/share/13834fdc-ea67-474f-b681-81e0eba737a7
    \item \textbf{0-4:} https://chat.openai.com/share/99b4e393-cb42-42af-b78d-d6109cce18a0
    \item \textbf{0-5:} https://chat.openai.com/share/896f7161-996d-4e72-bb5f-ae69cb5cb0da
\end{itemize}